\chapter{Automatyczne rozpoznawanie emocji}\label{ch:automatyczne-rozpoznawanie-emocji}

\section{Metody rozpoznawania emocji}\label{sec:metody-rozpoznawania-emocji}

Istnieje wiele sposobów, na podstawie których można wnioskować stan emocjonalny człowieka.
Pozwala to na wykorzystanie bardzo zróżnicowanych podejść, od oceny wyglądu, przez analizę zachowań, aż po pomiary aktywności elektrycznej w organizmie.
Z tego powodu do automatycznego rozpoznawania emocji wykorzystuje się różnoraką aparaturę oraz podejścia.

Poniżej znajduje się opis najczęściej stosowanych metod\cite{Varghese2015, Dzedzickis2020}, ich wady oraz zalety.

\subsection{Wyraz twarzy}\label{subsec:wyraz-twarzy}
Twarz