\chapter*{Wstęp}\label{ch:wstep}
\addcontentsline{toc}{chapter}{Wstęp}

Emocje są nieodłączną częścią ludzkiego życia.
Stanowią bardzo ważny element komunikacji niewerbalnej, wpływają na zachowanie i~postrzeganie świata.
Możliwość rozpoznawania emocji, reagowanie na nie oraz ich wywoływanie pozwoliłoby na rozwój w wielu dziedzinach.
Do najważniejszych należą robotyka, zwłaszcza interakcja człowiek-robot, marketing, szkolnictwo, przemysł rozrywkowy.
Doprowadziło to do powstania interdyscyplinarnej nauki pod nazwą informatyka afektywna (ang.~\textit{affective computing}), która łączy elementy informatyki, psychologii, neurologii, inżynierii i wielu innych.

Rozwój technologiczny pozwala na podejmowanie prób automatycznego rozpoznawania emocji przy pomocy systemów komputerowych.
Algorytmy uczenia maszynowego potrafią przetwarzać bardzo zróżnicowanie źródła informacji o emocjach.
Coraz mniejsze rozmiary sensorów i większa dokładność pozwalają na rejestrowanie przeróżnych parametrów: wyraz twarzy, sposób mowy, a nawet sygnały biofizyczne, takie jak EKG lub EEG\@.

Celem pracy jest wykorzystanie istniejącego zbioru zawierającego zapisy elektrokardiografii oraz reakcji skórno-galwanicznej i stworzenie systemu, który będzie w stanie dokonywać predykcji emocji.
W tym celu zaprojektowane zostanie kilka modeli, z wykorzystaniem różnych algorytmów uczenia maszynowego.
Następnie wykonane zostanie porównanie osiągniętych wyników.
