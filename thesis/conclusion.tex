\chapter*{Podsumowanie}\label{ch:podsumowanie}
\addcontentsline{toc}{chapter}{Podsumowanie}

Automatyczna predykcja emocji staje się coraz bardziej popularnym problemem, który może znaleźć zastosowanie w wielu dziedzinach ludzkiego życia.
Od dynamicznie rozwijającej się robotyki, przez naukę, aż po przemysł rozrywkowy, informacja o stanie emocjonalnym ma potencjał usprawnić sposób w~jaki prowadzimy interakcje z~danymi systemami.
W~niniejszej pracy przedstawiono podstawy automatycznej predykcji emocji.
Zaprezentowano źródła, na podstawie których istnieje możliwość predykcji stanów emocjonalnych.
Począwszy od tych, które można odbierać ludzkimi zmysłami, czyli wyrazy twarzy, mowa, gestykulacja, aż po wymagające specjalnej aparatury do odczytu — sygnałów biofizycznych.
Następnie przedstawiono możliwości reprezentacji informacji o~emocjach w~systemach komputerowych, podejścia oparte na wartościach liczbowych, jak i~kategoryzacji.

W~kolejnej części opisano uczenie maszynowe, które jest przodującym podejściem w~systemach automatycznej predykcji emocji.
Zaczęto od przedstawienia podstaw, po czym skupiono się na bardziej szczegółowym opisaniu dwóch popularnych algorytmów: maszyn wektorów nośnych oraz lasów losowych.
Następnie opisano podstawy sieci neuronowych, które wykazują duży potencjał i~są bardzo często wykorzystywane w~badaniach nad automatyczną predykcją emocji.

Na koniec zaproponowano kilka modeli, które przewidują emocje na podstawie elektrokardiografi oraz reakcji skórno-galwanicznej ze zbioru BIRAFFE2.
Niestety uzyskano wyniki, które nie są zadowalające, co obrazuje, jak skomplikowanym problemem jest próba rozpoznawania emocji, zwłaszcza na podstawie sygnałów biofizycznych.
Zaproponowano również podejścia, które mają potencjał poprawy wyników w~przyszłości.
