\chapter{Część praktyczna}\label{ch:czesc-praktyczna}

\section{Zbiór danych}\label{sec:zbior-danych}

W pracy wykorzystano gotowy zbiór danych o nazwie BIRAFFE2~\cite{Kutt2022}.
Zawiera on zapisy elektrokardiografii (EKG), reakcji skórno-galwanicznej (EDA), wyrazów twarzy i~ruchu dłoni, które zostały nagrane podczas prób wywołania emocji przez stymulanty audiowizualne i~specjalnie przygotowane gry komputerowe.
Dodatkowo w~zbiorze zawarto subiektywną ocenę stymulantów w~dwuwymiarowej przestrzeni przyjemności i~pobudzenia (ang.~\textit{valence, arousal}), wyniki testu osobowości opartego o tak zwaną wielką piątkę (ang.~\textit{big five}) oraz ankiety o doświadczeniu z~grami komputerowymi.
Dane pochodzą od 102 osób w~wieku od 18 do 26 lat, z~czego 33\% badanych to kobiety~\cite{Kutt2022}.
Rysunek~\ref{fig:BIRAFFE2-widget} przedstawia widżet, który wykorzystano do uzyskania wartości przyjemności i~pobudzenia.

\begin{figure}[h]
    \centering
    \includegraphics[scale=0.32]{BIRAFFE2_widget}
    \caption{Widżet użyty do subiektywnej oceny stymulantów. \textit{Źródło:~\cite{Kutt2022}}}
    \label{fig:BIRAFFE2-widget}
\end{figure}

Stymulanty audiowizualne prezentowane były w~dwóch turach, z~sesją gry komputerowej pomiędzy nimi.
Każdy stymulant prezentowany był przez 6 sekund, po czym badany miał 6 sekund na ocenę wywołanych emocji i~następowało kolejne 6 sekund przerwy~\cite{Kutt2022}.
Wizualne stymulanty wybrano ze zbioru IAPS~\cite{IAPS}, a~dźwiękowe ze zbioru IADS~\cite{IADS}.

W niniejszej pracy wykorzystano jedynie zapisy EKG oraz EDA z~obu tur prezentacji stymulantów.

\section{Przygotowanie danych}\label{sec:przygotowanie-danych}

\subsection{Oczyszczanie i ekstrakcja cech}\label{subsec:oczyszczanie-i-ekstrakcja-cech}

Cały system automatycznej predykcji emocji, opisany w~tej pracy, został napisany w~języku Python.
Aby oczyścić dane i~dokonać ekstrakcji cech (ang.~\textit{feature extraction}) użyto biblioteki NeuroKit\footnote{\url{https://neuropsychology.github.io/NeuroKit/}}~\cite{Neurokit}.
Zawiera ona wiele funkcji i~narzędzi pozwalających na pracę z~sygnałami biofizycznymi.

Dane zostały podzielone na okienka o długości 18 sekund, co odpowiada pojawieniu się pojedynczego stymulanta audiowizualnego, czasu na subiektywną ocenę emocji oraz przerwie przed kolejnym stymulantem.
Wartości odpowiadające treningowi nie były brane pod uwagę.

Sygnały EKG były poddawane oczyszczaniu funkcją \texttt{ecg\_clean()} z~wykorzystaniem metody zaproponowanej przez Pana i~Tompkinsa~\cite{Pan1985}.
Następnie znajdowano załamki R w~zespole QRS, wykorzystując metodę zaproponowaną w~tym samym artykule oraz funkcję \texttt{ecg\_peaks()}.
Na ich podstawie obliczano średnią częstotliwość występowania załamków funkcją \texttt{ecg\_rate()} oraz wartości związane ze zmiennością rytmu zatokowego (ang.~\textit{heart rate variability, HRV}) stosując \texttt{hrv\_time()} oraz \texttt{hrv\_frequency()}.

Podobnie jak EKG, sygnał EDA był na początku oczyszczany i~wydzielono z~niego tonic component, użyto do tego funkcję \texttt{eda\_process()}.
Następnie obliczono ilość wystąpień reakcji oraz ich średnią amplitudę wykorzystując funkcję \texttt{eda\_intervalrelated()}.
Kolejnym krokiem było obliczenie standardowego odchylenia dla tonic component.
Następnie wykorzystano \texttt{eda\_sympathetic()} aby uzyskać wartości związane z~sympathetic component, czyli sygnałem w~zakresie 0,0045 - 0,25 Hz~\cite{Posada2016}.
Na koniec obliczono autokorelację sygnału stosując \texttt{eda\_autocorr()}.

W kolejnym kroku zastosowano powyższe metody dla sygnałów z~przedziału od pierwszego do ostatniego stymulanta, które potraktowano jako średnią wartość, unikalną dla każdego badanego.
Następnie odejmowano wartości uzyskane w~każdym z~okienek od średniej danej osoby.
Miało to na celu uzyskanie danych o zmianie stanu badanego podczas oglądania stymulanta względem normy.

Dla każdego okienka przypisano odpowiadające mu dwie wartości uzyskane przez subiektywną ocenę.
Były to przyjemność emocji (ang.~\textit{valence}) oraz pobudzenie (ang.~\textit{arousal}) jakie wywołały.

\subsection{Grupowanie}\label{subsec:grupowanie}

Po uzyskaniu cech przeprowadzono proces grupowania (ang.~\textit{clustering}), w~celu zmiany problemu z~regresji do klasyfikacji dla kilku klas.
Wykorzystano do tego algorytm K-Means, który jest przykładem uczenia nienadzorowanego i~został zaproponowany przez Lloyda~\cite{Lloyd1982}.
Sama użyta funkcja \texttt{KMeans()} pochodzi z~biblioteki scikit-learn\footnote{\url{https://scikit-learn.org/}}~\cite{scikit-learn}.

Po obliczeniu centroidów oraz uzyskaniu grup ręcznie przypisano im emocje na podstawie modelu kołowego z~\cite{Kollias2019}, przedstawionego na obrazku~\ref{fig:emotions-wheel}.

\begin{figure}[h]
    \centering
    \includegraphics[scale=0.55]{emotions_wheel}
    \caption{Model kołowy użyty do przypisania emocj do grup. \textit{Źródło:~\cite{Kollias2019}}}
    \label{fig:emotions-wheel}
\end{figure}

W pracy stworzono modele dla ośmiu, sześciu i~czterech emocji.
Rysunek~\ref{fig:clusters-all} przedstawia pozyskane grupy.

\begin{figure}[h!]
    \begin{subfigure}{0.5\textwidth}
        \centering
        \includegraphics[scale=0.45]{clusters_8}
        \caption{8 emocji}
        \label{fig:clusters-8}
    \end{subfigure}
    \begin{subfigure}{0.5\textwidth}
        \centering
        \includegraphics[scale=0.45]{clusters_6}
        \caption{6 emocji}
        \label{fig:clusters-6}
    \end{subfigure}
    \begin{subfigure}{0.5\textwidth}
        \centering
        \includegraphics[scale=0.45]{clusters_4}
        \caption{4 emocje}
        \label{fig:clusters-4}
    \end{subfigure}
    \caption{Uzyskane grupy i~przypisane im emocje}
    \label{fig:clusters-all}
\end{figure}

\section{Wyniki}\label{sec:wyniki}

Do predykcji emocji wykorzystano trzy algorytmy: maszyny wektorów nośnych (SVM), lasy losowe (RFC) oraz sieć neuronową (NN).
Dla dwóch pierwszych użyto implementacji dostępnych w~scikit-learn, odpowiednio \texttt{SVC()} i~\texttt{RandomForestClassifier()}.
W przypadku sieci neuronowej wykorzystano natomiast bibliotekę TensorFlow\footnote{\url{https://tensorflow.org/}}~\cite{Tensorflow2015}.
Stworzono stosunkowo prostą wielowarstwową jednokierunkową sieć neuronową, składającą się z~gęstych warstw (ang.~\textit{dense layer}), czyli wszystkie neurony w warstwie były połączone ze wszystkimi neuronami w kolejnej warstwie.
Jako funkcję aktywacji wykorzystano ReLU, a~do nauki użyto SGD z~momentum.

Dane wejściowe były skalowane wykorzystując \texttt{StandardScaler()} z~scikit-learn, tak aby każda cecha posiadała średnią równą 0 oraz odchylenie standardowe równe 1.
Następnie dzielono dane na zbiór treningowy oraz zbiór testowy.
Zbiór testowy zawierał 10\% losowo wybranych wartości.
Sieć neuronowa wymagała dodatkowo zbioru walidacyjnego, który również zawierał 10\% losowo wybranych wartości.

Do oceny wyników modeli wykorzystano trzy miary.
Pierwsza z~nich, accuracy, mówi o~tym ile wartości zostało sklasyfikowanych poprawnie~\cite{Goodfellow2016}.
Pozostałe są oparte o~tak zwany F-score lub \(F_{1}\), który opiera się na dwóch miarach.
Precision, które opisuje, ile razy model zaklasyfikował przykład poprawnie, czyli ile razy nie popełnił błędu.
Recall, mówiące o tym ile wartości sklasyfikowano jako poprawne, co można interpretować jako umiejętność znalezienia wszystkich poprawnych wartości.
\(F_{1}\) można natomiast opisać wzorem: \(F = \frac{2pr}{p + r},\) gdzie \(p\)~to precision, a~\(r\)~to recall~\cite{Goodfellow2016}.

Miary te można jednak obliczać jedynie dla klasyfikatorów binarnych.
W~przypadku przewidywania wielu klas, oblicza się \(F_{1}\) osobno dla każdej.
Następnie można stosować wiele podejść.
W~Macro \(F_{1}\) uzyskuje się średnią z~wartości wszystkich klas.
Weighted \(F_{1}\) bierze dodatkowo pod uwagę Support, czyli ilość wystąpień klasy w~zbiorze.
Pozwala to na lepsze oszacowanie zbiorów, w~których ilość klas nie jest równa.

\subsection{Elektrokardiografia}\label{subsec:ekg}

Po wykonaniu powyższych kroków wybrano wartości związane z~elektrokardiografią oraz zmiennością rytmu zatokowego, co dało 29 cech.
Następnie dane ze zbiorów treningowych zostały przetworzone przez algorytmy.
Na koniec sprawdzono ich skuteczność na zbiorach treningowych.
Tabela~\ref{tab:table-ekg} zawiera uzyskane wyniki.

\begin{table}[h]
    \centering
    \begin{tabular}{||c||c||c|c|c||}
        \hline
        Ilość emocji & Model & Accuracy & Macro \(F_{1}\) & Weighted \(F_{1}\) \\ [0.5ex]
        \hline\hline
        \multirow{3}{4em}{Cztery} & SVM & 0.3113 & 0.1249 & 0.1547 \\
        \cline{2-5}
        & RFC & 0.3340 & 0.2743 & 0.2988 \\
        \cline{2-5}
        & NN & 0.3283 & 0.1390 & 0.1763 \\
        \hline\hline
        \multirow{3}{4em}{Sześć} & SVM & 0.2346 & 0.0690 & 0.0938 \\
        \cline{2-5}
        & RFC & 0.2763 & 0.2113 & 0.2389 \\
        \cline{2-5}
        & NN & 0.2857 & 0.2003 & 0.2363 \\
        \hline\hline
        \multirow{3}{4em}{Osiem} & SVM & 0.1892 & 0.0442 & 0.0646 \\
        \cline{2-5}
        & RFC & 0.2223 & 0.1601 & 0.1928 \\
        \cline{2-5}
        & NN & 0.2403 & 0.1592 & 0.2007 \\
        \hline
    \end{tabular}
    \caption{Wyniki systemu opartego na sygnale EKG.}
    \label{tab:table-ekg}
\end{table}

Dla czterech i~sześciu emocji najlepszym modelem były lasy losowe, natomiast dla ośmiu emocji lepsze wyniki osiągnęła sieć neuronowa.
Najgorszym modelem we wszystkich testach jest SVM, który jest również najmniej skomplikowany, przez co nie posiada on wystarczającej pojemności dla tak złożonych danych.

\subsection{Reakcja skórno-galwaniczna}\label{subsec:reakcja-skorno-galwaniczna}

Podobnie do EKG, po wyodrębnieniu cech związanych z reakcją skórno-galwaniczną uzyskano ich 11.
Tabela~\ref{tab:table-eda} zawiera uzyskane wyniki.

\begin{table}[h]
    \centering
    \begin{tabular}{||c||c||c|c|c||}
        \hline
        Ilość emocji & Model & Accuracy & Macro \(F_{1}\) & Weighted \(F_{1}\) \\ [0.5ex]
        \hline\hline
        \multirow{3}{4em}{Cztery} & SVM & 0.3113 & 0.1187 & 0.1478 \\
        \cline{2-5}
        & RFC & 0.3141 & 0.2848 & 0.3016 \\
        \cline{2-5}
        & NN & 0.3132 & 0.2032 & 0.2450 \\
        \hline\hline
        \multirow{3}{4em}{Sześć} & SVM & 0.2337 & 0.0631 & 0.0885 \\
        \cline{2-5}
        & RFC & 0.2886 & 0.2384 & 0.2672 \\
        \cline{2-5}
        & NN & 0.2621 & 0.1418 & 0.1790 \\
        \hline\hline
        \multirow{3}{4em}{Osiem} & SVM & 0.1892 & 0.0432 & 0.0643 \\
        \cline{2-5}
        & RFC & 0.2233 & 0.1828 & 0.2105 \\
        \cline{2-5}
        & NN & 0.2241 & 0.1335 & 0.1935 \\
        \hline
    \end{tabular}
    \caption{Wyniki systemu opartego o reakcję skórno-galwaniczną.}
    \label{tab:table-eda}
\end{table}

Wyniki są bardzo zbliżone do tych uzyskanych przez systemy bazujące na elektrokardiografi.
Lasy losowe oraz sieć neuronowa osiągnęły podobne wyniki, natomiast maszyna wektorów nośnych jest znacząco gorsza.
Dla czerech emocji najlepszy model (RFC) uzyskał gorsze accuracy niż system EKG, ale \(F_{1}\) były lepsze dla systemu opartego o~EDA\@.
W przypadku sześciu i~ośmiu emocji lepszy okazał się algorytm rozpoznający emocje na postawie reakcji skórno-galwanicznej.

\subsection{System wielomodalny}\label{subsec:system-wielomodalny}

Na koniec stworzono modele, oparte o cechy zarówno z~EKG, jak i~z~EDA, tworząc system wielomodalny.
Uzyskano dzięki temu 40 cech.
Tabela~\ref{tab:table-combined} przedstawia wyniki systemu wielomodalnego.

\begin{table}[h]
    \centering
    \begin{tabular}{||c||c||c|c|c||}
        \hline
        Ilość emocji & Model & Accuracy & Macro \(F_{1}\) & Weighted \(F_{1}\) \\ [0.5ex]
        \hline\hline
        \multirow{3}{4em}{Cztery} & SVM & 0.3122 & 0.1332 & 0.1641 \\
        \cline{2-5}
        & RFC & 0.3321 & 0.2848 & 0.3060 \\
        \cline{2-5}
        & NN & 0.3396 & 0.2202 & 0.2643 \\
        \hline\hline
        \multirow{3}{4em}{Sześć} & SVM & 0.2346 & 0.0690 & 0.0938 \\
        \cline{2-5}
        & RFC & 0.2933 & 0.2297 & 0.2606 \\
        \cline{2-5}
        & NN & 0.2990 & 0.2274 & 0.2591 \\
        \hline\hline
        \multirow{3}{4em}{Osiem} & SVM & 0.1911 & 0.0541 & 0.0783 \\
        \cline{2-5}
        & RFC & 0.2308 & 0.1810 & 0.2109 \\
        \cline{2-5}
        & NN & 0.2507 & 0.1935 & 0.2252 \\
        \hline
    \end{tabular}
    \caption{Wyniki systemu wielomodalnego.}
    \label{tab:table-combined}
\end{table}

Dla czterech i~ośmiu cech lepsze wyniki uzyskano dzięki zastosowaniu systemu wielomodalnego.
Natomiast dla sześciu emocji niewiele lepszy okazał się system oparty o~EDA\@.
Nie są to jednak bardzo duże różnice, wynoszą one około 1\%.
Tak jak w~systemach jednomodalnych, SVM dawały najgorsze wyniki, a~lasy losowe oraz sieci neuronowe były do siebie zbliżone.
Dla najbardziej skomplikowanego przypadku, czyli ośmiu emocji, sieć znowu uzyskała najlepsze wyniki.

\section{Podsumowanie wyników}\label{sec:podsumowanie-wynikow}

Niestety uzyskane wyniki są stosunkowo niskie.
Mimo to nadal są lepsze od losowego wybierania emocji, które wynosiłoby 25\% dla czterech emocji, 16,7\% dla sześciu i~12,5\% dla ośmiu.
Złożoność problemu najlepiej obrazują wyniki uzyskane przez prostsze maszyny wektorów nośnych, których \(F_{1}\) było 2–3 razy gorsze od bardziej skomplikowanych algorytmów.
Sygnały biofizyczne dodatkowo komplikują automatyczną predykcję, między innymi przez to, że są one niezależne od woli człowieka, oraz przez różnice między poszczególnymi osobami.
Nawet próby wywołania pewnych określonych emocji mogą skutkować odmiennymi reakcjami.
W~przypadku wyrazu twarzy lub mowy istnieje możliwość sztucznego przedstawiania emocji, lub ich wyolbrzymiania.
Nie da się uzyskać takiego efektu dla sygnałów biofizycznych.

Istnieje kilka podejść, które mają potencjał polepszenia wyników.
Zbiór danych zawiera zdjęcia wyrazów twarzy, które mogą zostać wykorzystane jako kolejne źródło informacji o~stanie emocjonalnym.
Mimo że nie są to sztucznie wyrażane emocje, a~naturalne reakcje badanych, systemy oparte o wyrazy twarzy uzyskują bardzo dobre wyniki oraz są jednym z~najpopularniejszych podejść w automatycznej predykcji stanów emocjonalnych.

Drugą możliwością jest dalsze eksperymentowanie z~sieciami neuronowymi.
W~niniejszej pracy uzyskiwały one najlepsze wyniki dla rozpoznawania ośmiu emocji, co pokazuje ich potencjał w skomplikowanych problemach.
Ilość hiperparametrów oraz potencjalne architektury sprawiają, że stworzenie najlepszej sieci neuronowej jest bardzo skomplikowane.
Możliwe jest wykorzystanie sieci konwolucyjnych, rekurencyjnych lub residual.

Kolejne podejścia mogą opierać się na wykorzystaniu odmiennego procesu przygotowania danych, na przykład przez ekstrakcję większej ilości cech lub użycie innych algorytmów zarówno do ekstrakcji, jak i~uczenia.
