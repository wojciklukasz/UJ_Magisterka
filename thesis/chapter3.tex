\chapter{Część praktyczna}\label{ch:czesc-praktyczna}

\section{Zbiór danych}\label{sec:zbior-danych}

W pracy wykorzystano gotowy zbiór danych o nazwie BIRAFFE2~\cite{Kutt2022}.
Zawiera on zapisy elektrokardiografii (EKG), reakcji skórno-galwanicznej (EDA), wyrazów twarzy i ruchu dłoni, które zostały nagrane podczas prób wywołania emocji przez stymulanty audiowizualne i specjalnie przygotowane gry komputerowe.
Dodatkowo w zbiorze zawarto subiektywną ocenę stymulantów w dwuwymiarowej przestrzeni przyjemności i pobudzenia (ang.~\textit{valence, arousal}), wyniki testu osobowości opartego o tak zwaną wielką piątkę (ang.~\textit{big five}) oraz ankiety o doświadczeniu z grami komputerowymi.
Dane pochodzą od 102 osób w wieku od 18 do 26 lat, z czego 33\% badanych to kobiety~\cite{Kutt2022}.

\begin{figure}[h]
    \centering
    \includegraphics[scale=0.35]{BIRAFFE2_widget}
    \caption{Widżet użyty do subiektywnej oceny stymulantów. \textit{Źródło:~\cite{Kutt2022}}}
    \label{fig:BIRAFFE2-widget}
\end{figure}

Stymulanty audiowizualne prezentowane były w dwóch turach, z sesją gry komputerowej pomiędzy turami.
Każdy stymulant prezentowany był przez 6 sekund, po czym badany miał 6 sekund na ocenę wywołanych emocji i następowało kolejne 6 sekund przerwy~\cite{Kutt2022}.
Wizualne stymulanty wybrano ze zbioru IAPS~\cite{IAPS}, a dźwiękowe ze zbioru IADS~\cite{IADS}.

W niniejszej pracy wykorzystano jedynie zapisy EKG oraz EDA z obu tur prezentacji stymulantów.

\section{Przygotowanie danych}\label{sec:przygotowanie-danych}

\subsection{Oczyszczanie i ekstrakcja cech}\label{subsec:oczyszczanie-i-ekstrakcja-cech}

Cały system automatycznej predykcji emocji, opisany w tej pracy, został napisany w języku Python.
Aby oczyścić dane i dokonać ekstrakcji cech (ang.~\textit{feature extraction}) użyto biblioteki NeuroKit\footnote{https://neuropsychology.github.io/NeuroKit/}~\cite{Neurokit}.
Zawiera ona wiele funkcji i narzędzi pozwalających na pracę z sygnałami biofizycznymi.

Dane zostały podzielone na okienka o długości 18 sekund, co odpowiada pojawieniu się pojedynczego stymulanta audiowizualnego, czasu na subiektywną ocenę emocji oraz przerwie przed kolejnym stymulantem.
Wartości odpowiadające treningowi nie były brane pod uwagę.

Sygnały EKG były poddawane oczyszczaniu funkcją \texttt{ecg\_clean()} z wykorzystaniem metody zaproponowanej przez Pana i Tompkinsa~\cite{Pan1985}.
Następnie znajdowano załamki R w zespole QRS, wykorzystując metodę zaproponowaną w tym samym artykule oraz funkcję \texttt{ecg\_peaks()}.
Na ich podstawie obliczano średnią częstotliwość występowania załamków funkcją \texttt{ecg\_rate()} oraz wartości związane ze zmiennością rytmu zatokowego (ang.~\textit{heart rate variability, HRV}) stosując \texttt{hrv\_time()} oraz \texttt{hrv\_frequency()}.

Podobnie jak EKG, sygnał EDA był na początku oczyszczany i wydzielono z niego tonic component, użyto do tego funkcję \texttt{eda\_process()}.
Następnie funkcją \texttt{eda\_intervalrelated()} obliczono ilość wystąpień reakcji oraz ich średnią amplitudę.
Kolejnym krokiem było obliczenie standardowego odchylenia dla tonic component.
Następnie wykorzystano \texttt{eda\_sympathetic()} aby uzyskać wartości związane z sympathetic component, czyli wartościami w zakresie 0,0045 - 0,25 Hz~\cite{Posada2016}.
Na koniec obliczono autokorelację sygnału stosując \texttt{eda\_autocorr()}.

W kolejnym kroku zastosowano powyższe metody dla sygnałów z przedziału od pierwszego do ostatniego stymulanta, które potraktowano jako średnią wartość, unikalną dla każdego badanego.
Następnie odejmowano wartości uzyskane w każdym z okienek od średniej danej osoby.
Miało to na celu uzyskanie danych o zmianie stanu badanego podczas oglądania stymulanta względem normy.

Dla każdego okienka przypisano odpowiadające mu dwie wartości uzyskane przez subiektywną ocenę.
Były to przyjemność emocji (ang.~\textit{valence}) oraz pobudzenie (ang.~\textit{arousal}) jakie wywołały.

\subsection{Grupowanie}\label{subsec:grupowanie}

Po uzyskaniu cech przeprowadzono proces grupowania (ang.~\textit{clustering}), w celu zmiany problemu z regresji do klasyfikacji dla kilku klas.
Wykorzystano do tego algorytm K-Means, który jest przykładem uczenia nienadzorowanego i został zaproponowany przez Lloyda~\cite{Lloyd1982}.
Sama użyta funkcja \texttt{KMeans()} pochodzi z biblioteki scikit-learn\footnote{https://scikit-learn.org/}~\cite{scikit-learn}.

Po obliczeniu centroidów oraz uzyskaniu grup ręcznie przypisano im emocje na podstawie modelu kołowego z~\cite{Kollias2019}.
W pracy stworzono modele dla 8, 6 i 4 emocji.

\begin{figure}[h]
    \centering
    \includegraphics[scale=0.5]{emotions_wheel}
    \caption{Model kołowy użyty do przypisania emocj do grup. \textit{Źródło:~\cite{Kollias2019}}}
    \label{fig:emotions-wheel}
\end{figure}

\begin{figure}[h!]
    \begin{subfigure}{0.5\textwidth}
        \centering
        \includegraphics[scale=0.45]{clusters_8}
        \caption{8 emocji}
        \label{fig:clusters-8}
    \end{subfigure}
    \begin{subfigure}{0.5\textwidth}
        \centering
        \includegraphics[scale=0.45]{clusters_6}
        \caption{6 emocji}
        \label{fig:clusters-6}
    \end{subfigure}
    \begin{subfigure}{0.5\textwidth}
        \centering
        \includegraphics[scale=0.45]{clusters_4}
        \caption{4 emocje}
        \label{fig:clusters-4}
    \end{subfigure}
    \caption{Uzyskane grupy i przypisane im emocje.}
    \label{fig:clusters-all}
\end{figure}

\section{Wyniki}\label{sec:wyniki}
Porównanie wyników różnych modeli scikit-learn i TensorFlow oraz modalności, kilka tabelek.

\begin{table}[h!]
    \centering
    \begin{tabular}{||c c c c||}
        \hline
        Col1 & Col2 & Col2 & Col3 \\ [0.5ex]
        \hline\hline
        1 & 6 & 87837 & 787 \\
        2 & 7 & 78 & 5415 \\
        3 & 545 & 778 & 7507 \\
        4 & 545 & 18744 & 7560 \\
        5 & 88 & 788 & 6344 \\ [1ex]
        \hline
    \end{tabular}
    \caption{Table to test captions and labels.}
    \label{tab:table1}
\end{table}
